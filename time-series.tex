\documentclass{article}
\usepackage{amsmath,mathtools}
\title{High-Resolution Time Series Visualization}
\author{Caleb Reach}
\date{\today}
\let\V\boldsymbol
\begin{document}
\maketitle
Suppose that we wish to visualize a bandlimited signal $f(t)$.  We can define the visualization as
\[
u(\V x) = \int_{-\infty}^{\infty} \delta\left(\V x - \begin{bmatrix}t\\f(t)\end{bmatrix}\right)\,dt.
\]
In practice, we wish to compute a sampled image.  Because $u(\V x)$ is not bandlimited, it should not be sampled directly.  Therefore, we compute
\[
v = g*u
\]
where $g$ is the impulse response of a anti-aliasing lowpass filter.  The point in the sampled image with integer coordinates $(x,y)$ is then given by
\[
v\left(\begin{bmatrix}xT_x\\yT_y\end{bmatrix}\right)
\]
for horizontal sampling rate $T_x$ and vertical sampling rate $T_y$.

Because $f(t)$ is bandlimited, it is well-approximated by sampling followed by linear interpolation, provided that the sampling rate is sufficiently higher than twice the highest frequency in $f(t)$.  Let $T_l$ be the sampling period given by this sampling rate.  Therefore,
\[
f \approx l*(\mathrm{III}_{T_l} f),
\]
where $l$ is the impulse response of a linear interpolation filter, and thus
\begin{align*}
v(\V x) &\approx g(\V x)*\int_{-\infty}^{\infty} \delta\left(\V x - \begin{bmatrix}t\\(l*(\mathrm{III}_{T_l} f))(t)\end{bmatrix}\right)\,dt\\
&= g(\V x)*\sum_{i=-\infty}^{\infty}\frac{1}{T_l}\int_{0}^{1}\delta\left(\V x - \begin{bmatrix}iT_l + tT_l\\(1 - t)f(iT_l) + tf((i+1)T_l)\end{bmatrix}\right)\,dt.
\end{align*}
Notice that when $T_l \ll T_x$, the lines constructed by linear interpolation are nearly vertical (or else nearly points) in the sampled image.  Therefore, it is reasonable\footnote{This can be expressed as convolving $u(\V x)$ with a box filter of width $T_l$ in the horizontal direction and then multiplying with an impulse train of period $T_l$ in the horizontal direction.  This process introduces aliasing and imaging artifacts, but because $T_l \ll T_x$, these artifacts are concentrated at high-frequencies and are consequently almost entirely subsequently filtered out by $g$.} to approximate subexpression $iT_l + tT_l$ by $(i + 0.5)T_l$.  We thus arrive at the approximation
\[
\hat v(\V x) = g(\V x)*\sum_{i=-\infty}^{\infty}\frac{1}{T_l}\int_{0}^{1}\delta\left(\V x - \begin{bmatrix}(i + 0.5)T_l\\(1 - t)f(iT_l) + tf((i+1)T_l)\end{bmatrix}\right)\,dt.
\]
If we define
\begin{align*}
g_I\left(\begin{bmatrix}x\\y\end{bmatrix}\right) &= \int_{-\infty}^y g\left(\begin{bmatrix}x\\t\end{bmatrix}\right)\,dt\\
\V \alpha_i &= \begin{bmatrix}\tau_i\\(1 - t)f(iT_l))\end{bmatrix}\\
\V \beta_i &= \begin{bmatrix}\tau_i\\tf((i+1)T_l)\end{bmatrix}\\
\tau_i &= (i+0.5)Tl,
\end{align*}
then we can rewrite $\hat v$ as
\begin{align*}
\hat v(\V x) &= g_I(\V x)*\sum_{i=-\infty}^{\infty}\frac{\delta(\V x - \V \alpha_i) - \delta(\V x - \V \beta_i)}{T_l\|\V\beta_i - \V\alpha_i\|}\\
&= \sum_{i=-\infty}^{\infty}\frac{g_I(\V x - \V \alpha_i) - g_I(\V x - \V \beta_i)}{T_l\|\V\beta_i - \V\alpha_i\|}.
\end{align*}
Notice that if $g(\V x)$ is compactly supported then
\[
g_D = g_I(\V x) - g_I\left(\V x - \begin{bmatrix}0\\T_y\end{bmatrix}\right)
\]
is also compactly supported.  We can thus produce a line image by adding scaled copies of $g_D$ at the start and ends of each line in the linear interpolation and then calculating the cumulative sum in the y direction once at the end, in which case drawing a line is a constant-time operation.  To improve uniformity, each sampling of $g_D$ should be normalized.

If $\|\V\beta_i - \V\alpha_i\|$ is close to zero (i.e. when neighboring samples have similar values), numerical stability issues arise.  We can approximate this by taking the average of two copies of $g$.  The summand of $\hat v(\V x)$ can be seen as the convolution of $g$ centered at the midpoint between $\V \alpha_i$ and $\V \beta_i$ with a vertical box filter kernel.  This box kernel can be approximated by the average of two dirac delta functions.  The second moment of the box function with width $w$ is $w^2/3$.  The second moment of the average of two dirac delta functions located at $\pm a$ is $a^2$.  Solving $w^2/3 = a^2$ for $a$ gives $\pm w/\sqrt{3}$.  Looking in the frequency domain reveals that $\delta(t \pm w/\sqrt{3})$ well-approximates the box kernel for frequencies with periods sufficiently longer than $w$.


% We approximate the frequency response of the boxcar function by the frequency response of two summed deltas.  We only care about the accuracy of the approximation below the cutoff of the anti-aliasing filter.  Matching standard deviations of uniform distribution and .
% Therefore, it is useful to use the following numerically-stable approximation:
% \[
% \hat v(\V x) \approx \sum_{i=-\infty}^{\infty}
% \begin{dcases}
% \frac{g_I(\V x - (\V \alpha_i + \V \beta_i)/2)}{T_l} & \|\V\beta_i - \V\alpha_i\| < \epsilon\\
% \frac{g_I(\V x - \V \alpha_i) - g_I(\V x - \V \beta_i)}{T_l\|\V\beta_i - \V\alpha_i\|} & \text{otherwise}
% \end{dcases}
% \]
\end{document}
